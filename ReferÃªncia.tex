\chapter{Revisão Bibliográfica}

\section{Dijkstra}

	Em \cite{algo} é apresentado o algoritmo de dijkistra, sendo este utilizado para obter o menor caminho de um vértice de origem até cada um dos outros vértices do  grafo, onde G é um grafo simples, caso este não seja simples, é necessário torná-lo. O artigo ressalta que ao contrário do algorítimo Bellman-Ford, Bellman-Ford impõe restrições sobre o sinal do peso das arestas, criando uma solução menor, dependente dos vértices de início e final.
	
	Ainda em \cite{algo} é resolvido um pequeno exemplo com o algoritmo de dijkistra que envolve distância entre cidades, mostrando o menor caminho entre elas. Esse exemplo foi feito passo a passo, explicando minunciosamente como funciona esse algoritmo. Também ressalta que este só pode ser utilizado em grafos ponderados e unicamente com pesos positivos, calculando a distância entre uma cidade e todas as outras, diferentemente do Algoritmo de Floyd que calcula a distância entre todas as cidades.
	
	Em \cite{barros2007algoritmo} há a apresentação do Algoritmo de dijkistra e a explicação do algoritmo passo a passo, feita de forma diferente do \cite{algo} pois este é feito de forma mais mecânica, com um exemplo mecânico. Apenas com uma tabela e como o algoritmo funciona e seus passos. 
	
	Em \cite{barros2007algoritmo} também é dita algumas aplicações, indo de uma cadeia de produção, até o clássico problema do carteiro que não pode passar duas vezes na mesma rua. Qualquer grafo simples que possua a matriz de pesos definida pode ser submetida à proposta de dijkistra.
	
\section{Bellman-Ford}	
	Em \cite{ford} é apresentado o algorítimo de Bellman-Ford, sendo este utilizado para obter o menor caminho de um nodo de origem até cada um dos outros nodos de  G, onde G é um dígrafo(grafo orientado) com arestas ponderadas. O artigo ressalta que ao contrário do algorítimo Dijkstra, Bellman-Ford não impõe restrições sobre o sinal do peso das arestas, criando uma solução mais genérica.

	Ainda em \cite{ford}, algumas das principais aplicações são mostradas, Protocolos de Roteamento Vetor-Distância e Problema "Triangular Arbitrage", útil para problemas da área de economia e investimento.

	Mesmo com estruturas similares,algumas diferenças são vistas entre Dijkstra e Ford em \cite{barros2007algoritmo} e \cite{ford}, enquanto o primeiro busca exaustivamente o nodo com menor peso ainda não computado,o último usa o procedimento de relaxamento V - 1 vezes, onde V representa a quantidade de vértices em G.