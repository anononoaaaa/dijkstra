\documentclass[
	% -- opções da classe memoir --
	12pt,				% tamanho da fonte
	openright,			% capítulos começam em pág ímpar (insere página vazia caso preciso)
	oneside,			% para impressão em recto e verso. Oposto a oneside
	a4paper,			% tamanho do papel. 
	% -- opções da classe abntex2 --
	%chapter=TITLE,		% títulos de capítulos convertidos em letras maiúsculas
	%section=TITLE,		% títulos de seções convertidos em letras maiúsculas
	%subsection=TITLE,	% títulos de subseções convertidos em letras maiúsculas
	%subsubsection=TITLE,% títulos de subsubseções convertidos em letras maiúsculas
	% -- opções do pacote babel --
	english,			% idioma adicional para hifenização
	french,				% idioma adicional para hifenização
	spanish,			% idioma adicional para hifenização
	brazil,				% o último idioma é o principal do documento
	]{abntex2}


% ---
% PACOTES
% ---

% ---
% Pacotes fundamentais 
% ---
\usepackage{lmodern}			% Usa a fonte Latin Modern
\usepackage[T1]{fontenc}		% Selecao de codigos de fonte.
\usepackage[utf8]{inputenc}		% Codificacao do documento (conversão automática dos acentos)
\usepackage{indentfirst}		% Indenta o primeiro parágrafo de cada seção.
\usepackage{color}				% Controle das cores
\usepackage{graphicx}			% Inclusão de gráficos
\usepackage{microtype} 			% para melhorias de justificação
\usepackage{placeins}
% ---

% ---
% Pacotes adicionais, usados no anexo do modelo de folha de identificação
% ---
\usepackage{multicol}
\usepackage{multirow}
% ---
	
% ---
% Pacotes adicionais, usados apenas no âmbito do Modelo Canônico do abnteX2
% ---
\usepackage{lipsum}				% para geração de dummy text
% ---

% ---
% Pacotes de citações
% ---
\usepackage[brazilian,hyperpageref]{backref}	 % Paginas com as citações na bibl
\usepackage[alf]{abntex2cite}	% Citações padrão ABNT

% --- 
% CONFIGURAÇÕES DE PACOTES
% --- 

% ---
% Configurações do pacote backref
% Usado sem a opção hyperpageref de backref
\renewcommand{\backrefpagesname}{Citado na(s) página(s):~}
% Texto padrão antes do número das páginas
\renewcommand{\backref}{}
% Define os textos da citação
\renewcommand*{\backrefalt}[4]{
	\ifcase #1 %
		Nenhuma citação no texto.%
	\or
		Citado na página #2.%
	\else
		Citado #1 vezes nas páginas #2.%
	\fi}%
% ---
%redefine a capa
\renewcommand{\imprimircapa}{%
	\begin{capa}%
		\center
		\ABNTEXchapterfont\Large \textbf{UNIVERSIDADE FEDERAL DE SERGIPE}
		\\
		\vspace*{1cm}
		{\ABNTEXchapterfont\large\imprimirautor}
		\vfill
		\begin{center}
			\ABNTEXchapterfont\bfseries\LARGE\imprimirtitulo
		\end{center}
		\vfill
		\large\imprimirlocal \\
		\large\imprimirdata
		\vspace*{1cm}
	\end{capa}
}
% ---
% Informações de dados para CAPA e FOLHA DE ROSTO
% ---
\titulo{Grafos \& Garfos}
\autor{Iuri rodrigo ferreira Alves da silva\\Gregory Medeiros Melgaço Pereira\\Raul Rodrigo Silva de Andrade \\ Rafael Castro Nunes \\ Ruan Robert Bispo dos Santos \\ Vítor do Bomfim Almeida Carvalho}
\local{São Cristóvão,SE}
\data{\today}
\instituicao{%
  Universidade Federal De Sergipe
  \par
  Faculdade de Engenharia Eletrônica
  \par
  Redes e Comunicações}
\tipotrabalho{Relatório técnico}
% O preambulo deve conter o tipo do trabalho, o objetivo, 
% o nome da instituição e a área de concentração 
\preambulo{Relatório em conformidade com as normas ABNT(pra falar a vdd n sei oq escrever aqui)}
% ---

% ---
% Configurações de aparência do PDF final

% alterando o aspecto da cor azul
\definecolor{blue}{RGB}{41,5,195}

% informações do PDF
\makeatletter
\hypersetup{
     	%pagebackref=true,
		pdftitle={\@title}, 
		pdfauthor={\@author},
    	pdfsubject={\imprimirpreambulo},
	    pdfcreator={LaTeX with abnTeX2},
		pdfkeywords={abnt}{latex}{abntex}{abntex2}{relatório técnico}, 
		colorlinks=true,       		% false: boxed links; true: colored links
    	linkcolor=blue,          	% color of internal links
    	citecolor=blue,        		% color of links to bibliography
    	filecolor=magenta,      		% color of file links
		urlcolor=blue,
		bookmarksdepth=4
}
\makeatother
% --- 

% --- 
% Espaçamentos entre linhas e parágrafos 
% --- 

% O tamanho do parágrafo é dado por:
\setlength{\parindent}{1.3cm}

% Controle do espaçamento entre um parágrafo e outro:
\setlength{\parskip}{0.2cm}  % tente também \onelineskip

% ---
% compila o indice
% ---
\makeindex
% ---

% ----
% Início do documento
% ----
\begin{document}

% Seleciona o idioma do documento (conforme pacotes do babel)
%\selectlanguage{english}
\selectlanguage{brazil}

% Retira espaço extra obsoleto entre as frases.
\frenchspacing 

% ----------------------------------------------------------
% ELEMENTOS PRÉ-TEXTUAIS
% ----------------------------------------------------------

% ---
% Capa
% ---
\imprimircapa
% ---

% ---
% Folha de rosto
% (o * indica que haverá a ficha bibliográfica)
% ---
\imprimirfolhaderosto*
% ---

% ---
% Agradecimentos
% ---
\begin{agradecimentos}
	
	O agradecimento principal é direcionado a Ruan, por ter feito o trabalho todo.
	
	Agradecimento especial ao querido professor felix, por sabotar a prova de redes.
	
	a vida é bonita é bonita
	
\end{agradecimentos}
% ---

% RESUMO
% resumo na língua vernácula (obrigatório)
\setlength{\absparsep}{18pt} % ajusta o espaçamento dos parágrafos do resumo
\begin{resumo}
	Colocar aqui um resumo bem legal aqui.
	
	\noindent
	\textbf{Palavras-chaves}: Grafos. Dijkstra.
\end{resumo}
% ---

%lista de ilustrações
% ---
\pdfbookmark[0]{\listfigurename}{lof}
\listoffigures*
\pagebreak
% ---

%lista de tabelas
\pdfbookmark[0]{\listtablename}{lot}
\listoftables*
\pagebreak
% ---

%lista de abreviaturas e siglas
\begin{siglas}
	\item[Rn] Ruan
	\item[Ru] Raul
\end{siglas}
% ---

%lista de símbolos
\begin{simbolos}
	\item[$ \Gamma $] Letra grega Gama
	\item[$ \Lambda $] Lambda
	\item[$ \zeta $] Letra grega minúscula zeta
	\item[$ \in $] Pertence
\end{simbolos}
% ---

% sumario
\pdfbookmark[0]{\contentsname}{toc}
\tableofcontents*
\pagebreak
% ---

%introdução
\chapter*[Introdução]{Introdução}
\addcontentsline{toc}{chapter}{Introdução}

No mundo globalizado e capitalista que vivemos é cada vez mais necessário caminhos ou métodos para se decidir qual a melhor alternativa ou caminho a ser usado visando uma maior economia, , seja ela na produção do meio industrial, nas conversões de moedas ou até melhores caminhos a serem tomados numa viagem. 

Foi pensando nesses tipos de problemas que surgiu a Teoria dos Grafos em 1736 com Leonhard Euler (1707 – 1783). Um grafo é um conjunto de vértices e um conjunto de arestas que ligam pares de vértices distintos (com nunca mais que uma aresta a ligar qualquer par de vértices).  Leonhard resolveu um quebra-cabeça local da cidade Königsberg onde vivia, hoje conhecido como O Problema das Pontes de Königsberg. Esse problema consistia em partindo de um ponto inicial, voltar a esse mesmo ponto passando exatamente uma vez por cada ponte da cidade. Euler mostrou que isso só seria possível se cada porção de terra estivesse ligada a um número par de pontes, e , como isso não acontecia na sua cidade, não era possível.  Outro exemplo nesse mesmo sentido é o mostrado em \cite{costalonga2012grafos}, em que um carteiro sempre tentará caminhar a menor distancia possível, assim, para isso acontecer, ele tem de agir de tal forma a passar um número mínimo de vezes em determinada rua, otimizando assim seu trabalho. Ao longo da história outros problemas similares ao de Euler surgiram, como por exemplo ‘O Problema das Quatro Cores’ . Esse problema surgiu em 1852 e somente em 1976 utilizando métodos computacionais chegou-se a sua solução. 

A Teoria dos Grafos ganhou muita visibilidade já no século passado, pois como dito em \cite{ da2011teoria} no desenvolvimento matemático  assim como nas suas aplicações, foi dada grande importância a essa teoria. Com o passar do tempo grafos foram utilizados em várias áreas do conhecimento, desde a Economia até a Biologia. 

Com esse grande destaque, se tornou necessária a criação de algoritmos que agilizassem suas soluções. Entre os mais conhecidos estão o de Dijkstra, de Bellman-Ford e o de Floyd-Warshall.



	
\phantompart
%-

%Revisão Bibliografica
\include{Referência}
\phantompart
%-
\chapter{Objetivos}
\chapter{Fundamentação Teórica}

\section{Grafos}

O conceito de grafo é um conceito simples, porém muito amplo, que trata da relação de conexão entre elementos nos mais diversos problemas matemáticos. Podemos então definir grafo como a união de um conjunto de vértices e um conjunto de arestas. Geralmente os grafos tem a seguinte notação:

\begin{equation}
G = (V,A)
\end{equation}

Onde V representa o conjunto de vértices e A representa o conjunto de arestas.

\subsection{Vértices e Arestas }

Os vértices, também chamados de nós, são os elementos de um conjunto.   O número de elementos deste conjunto representa a ordem da estrutura. O conjunto de arestas representa as ligações entre os elementos do conjunto de vértices.

A depender do problema a ser estudado, estes elementos assumem significados diferentes. Os vértices podem significar pessoas, localizações geográficas, hosts, entre outros. Já as arestas representam então as relações entre esses elementos, desde distâncias físicas entre cidades, até relações de amizade ou afetividade entre pessoas.

\subsection{Antecessor, Sucessor e Vizinho}

Vértices vizinhos são aqueles que são ligados por uma aresta ou arco. Quando este vértice está na extremidade inicial da aresta ele é chamado de antecessor, quando está na extremidade final, é chamado de vértice sucessor.

\subsection{Grafos Orientados ou Não Orientados}

Em um grafo não orientado, a ligação entre dois vértices é feita através de uma linha, chamada de aresta. Já em um grafo orientado, a ligação é feita através de uma seta, chamada de arco, e esta representa o sentido correspondente desta ligação.

\begin{figure}[h!]
	\centering
	\includegraphics[scale=0.5]{img1.jpeg}
	\caption{Grafo Orientado}
	\label{img1}
\end{figure}
\begin{figure}[h!]
	\centering
	\includegraphics[scale=0.5]{img2.jpeg}
	\caption{Grafo Não Orientado}
	\label{img2}
\end{figure}

\section{Representação de um grafo}

Para melhor visualização de um grafo é usada a representação como nas figuras \ref{img1} e \ref{img2}, com a identificação dos vértices entre círculos e as arestas ou arcos como linhas, porém para fins de cálculo os grafos são associados a matrizes. As matriz mais comumente utilizada é a matriz de adjacência, apesar de não ser a mais econômica computacionalmente.

\subsection{Matriz de Adjacência}

A matriz de adjacência é uma matriz booleana, quando utilizada em grafos, de tamanho n x n onde n é o número de vértices do grafo. Chamando de $A=a_{ij}$ a matriz adjacência, onde i  e j representam o número de identificação do vértice, por exemplo, para $a_{21}$ teremos nesta posição o valor correspondente a aresta que liga o vértice dois ao vértice um. A matriz é preenchida pela seguinte condição:

$a_{ij}=1$, se existe ligação entre os vértices i e j

$a_{ij}=0$, se não existe ligação entre os vértices i e j

Exemplo:

\begin{figure}[h!]
	\centering
	\includegraphics[scale=0.5]{img3.jpeg}
	\caption{Grafo Sem Peso}
	\label{img3}
\end{figure}

\begin{table}[h!]
	\centering
	\label{my-label}
	\begin{tabular}{ccccc}
		& A & B & C                      & D                      \\
		\multicolumn{1}{c|}{A} & 0 & 1 & \multicolumn{1}{c}{1} & \multicolumn{1}{c|}{1} \\
		\multicolumn{1}{c|}{B} & 1 & 0 & \multicolumn{1}{c}{0} & \multicolumn{1}{c|}{0} \\
		\multicolumn{1}{c|}{C} & 1 & 0 & \multicolumn{1}{c}{0} & \multicolumn{1}{c|}{1} \\
		\multicolumn{1}{c|}{D} & 1 & 0 & \multicolumn{1}{c}{1} & \multicolumn{1}{c|}{0}
	\end{tabular}
	\caption{Matriz Adjacência sem Peso}
\end{table}
\FloatBarrier
Caso as arestas contenham peso, a matriz de adjacência deve ser preenchida com o peso correspondente e onde não houver conexão entre os vértices, preencher com um valor que não possa ser considerado peso, como zero ou infinito.

Exemplo:

\begin{figure}[h!]
	\centering
	\includegraphics[scale=0.5]{img4.jpeg}
	\caption{Grafo Com Peso}
	\label{img4}
\end{figure}

\begin{table}[h!]
	\centering
	\label{my-label}
	\begin{tabular}{ccccc}
		& A & B & C                      & D                      \\
		\multicolumn{1}{c|}{A} & 0 & 4 & \multicolumn{1}{c}{6} & \multicolumn{1}{c|}{3} \\
		\multicolumn{1}{c|}{B} & 4 & 0 & \multicolumn{1}{c}{0} & \multicolumn{1}{c|}{0} \\
		\multicolumn{1}{c|}{C} & 6 & 0 & \multicolumn{1}{c}{0} & \multicolumn{1}{c|}{2} \\
		\multicolumn{1}{c|}{D} & 3 & 0 & \multicolumn{1}{c}{2} & \multicolumn{1}{c|}{0}
	\end{tabular}
	\caption{Matriz Adjacência com Peso}
\end{table}
\FloatBarrier
Uma observação importante é que, em grafos não direcionados, a matriz de adjacência é simétrica em relação a diagonal principal.

\section{Aplicação para o Problema do Menor Caminho}

Uma das aplicações mais comuns de grafos é para a resolução do problema de menor caminho, definindo um vértice de partida e um de chegada em um grafo, determinar qual caminho apresenta o menor custo para percorrer este caminho, considerando os pesos contidos nas arestas. Devido a importância da resolução deste tipo de problema, foram desenvolvidos algoritmos capazes de retornar como resultado o menor caminho dentro de um grafo.

\subsection{Algoritmo de Dijkstra}

Para utilizarmos o algoritmo de Dijkstra o primeiro passo é determinar o vértice de origem e o vértice de destino. Determinados a origem e o destino é feita uma análise através de interações vértice a vértice, buscando o menor, ou menores caminhos possíveis.

Partindo da origem, compara-se o custo de cada conexão ligada ao vértice do momento, para a primeira interação este vértice será a origem. Segue-se então para o próximo vértice, que apresentar o menor custo e o vértice antecessor é marcado. No novo vértice do momento é somado os custos para a chegada até ele e busca-se as conexões do vértice do momento com vértices não marcados, seguindo para o que apresente menor custo. Essa interação se repete até chegar no vértice de destino.

É importante salientar que este algoritmo não aceita custos (pesos) negativos.

\subsection{Algoritmo de Floyd- Warshall}

Este algoritmo é matricial e aceita custos negativos, mas a presença de ciclos negativos, ou seja, presença de mais de um valor negativo em uma malha de arestas, exige cuidado e conferência, já que o resultado pode não condizer com o real.
	
O algoritmo calcula todos os menores custos de todas as origens para todos os destinos, através de um número de interações finito e igual ao número de vértices do grafo. O algoritmo é inicializado com uma matriz onde os elementos são apenas as distâncias diretas entre os vértices, a diagonal principal, ou seja a distância entre os vértices e eles mesmos é zero e os vértices que não tem ligação direta, o elemento da matriz será infinito.

Após a montagem da matriz inicial, cada interação busca as menores distâncias entre os pontos, passando pelo vértice de número igual ao número da interação. Por exemplo, a terceira interação busca as menores distâncias entre os vértices, passando pelo vértice de número 3. Em cada interação, caso a distância seja menor do que a que já obtemos na matriz, o valor menor substitui o atual. Ao final das interações temos todas as menores distâncias entre os vértices.
\phantompart
\chapter{Formulação do problema}
\phantompart
\chapter{Resultados Obtidos}
\phantompart
\chapter{Conclusão}
\phantompart

\cite{algo}
\cite{cardoso2005teoria}
\cite{coelho2013teoria}
\cite{costa2011teoria}
\cite{costalonga2012grafos}
\cite{deteoria}
\cite{feofiloff2011introduccao}


\bibliography{biblio}
%\bibliographystyle{ieeetr}
\end{document}
