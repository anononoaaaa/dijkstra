\chapter*[Introdução]{Introdução}
\addcontentsline{toc}{chapter}{Introdução}

No mundo globalizado e capitalista que vivemos é cada vez mais necessário caminhos ou métodos para se decidir qual a melhor alternativa ou caminho a ser usado visando uma maior economia, , seja ela na produção do meio industrial, nas conversões de moedas ou até melhores caminhos a serem tomados numa viagem. 

Foi pensando nesses tipos de problemas que surgiu a Teoria dos Grafos em 1736 com Leonhard Euler (1707 – 1783). Um grafo é um conjunto de vértices e um conjunto de arestas que ligam pares de vértices distintos (com nunca mais que uma aresta a ligar qualquer par de vértices).  Leonhard resolveu um quebra-cabeça local da cidade Königsberg onde vivia, hoje conhecido como O Problema das Pontes de Königsberg. Esse problema consistia em partindo de um ponto inicial, voltar a esse mesmo ponto passando exatamente uma vez por cada ponte da cidade. Euler mostrou que isso só seria possível se cada porção de terra estivesse ligada a um número par de pontes, e , como isso não acontecia na sua cidade, não era possível.  Outro exemplo nesse mesmo sentido é o mostrado em \cite{costalonga2012grafos}, em que um carteiro sempre tentará caminhar a menor distancia possível, assim, para isso acontecer, ele tem de agir de tal forma a passar um número mínimo de vezes em determinada rua, otimizando assim seu trabalho. Ao longo da história outros problemas similares ao de Euler surgiram, como por exemplo ‘O Problema das Quatro Cores’ . Esse problema surgiu em 1852 e somente em 1976 utilizando métodos computacionais chegou-se a sua solução. 

A Teoria dos Grafos ganhou muita visibilidade já no século passado, pois como dito em \cite{ da2011teoria} no desenvolvimento matemático  assim como nas suas aplicações, foi dada grande importância a essa teoria. Com o passar do tempo grafos foram utilizados em várias áreas do conhecimento, desde a Economia até a Biologia. 

Com esse grande destaque, se tornou necessária a criação de algoritmos que agilizassem suas soluções. Entre os mais conhecidos estão o de Dijkstra, de Bellman-Ford e o de Floyd-Warshall.